Das Buchholzrelais ist ein mechanisches Schutzgerät für flüssigkeitsisolierte Transformatoren, Drossel- und Erdschlussspulen. Es wird in der Rohrleitung zwischen dem Kessel und dem Ausdehnungsgefäß des zu schützenden Geräts verbaut. Für eine verlässliche Funktionsweise, muss die Rohrleitung leicht geneigt sein. Dadurch sammelt sich das im Schutzgerät aufsteigende Gas im Buchholzrelais. Über die Gasansammlung detektiert das Buchholzrelais kleinere Fehler, die über einen längeren Zeitraum entstanden sind. Zu diesen kleineren Fehlern zählen laut Messko\cite{messkoManual} energieschwache Teilentladungen, Kriechströme, lokale und temporäre Überhitzungen, sowie laut dem IEEE\cite[][S. 32]{ieeeGuide} eine durch Überlastung beschleunigte Alterung. Größere Fehler, wie Lichtbogenentladungen oder stromstarke Überschläge, im Transformator werden anhand der ruckartig ansteigenden Strömungsgeschwindigkeit festgestellt.
\cite{embKatalog},
\cite[][S. 32ff]{abbManual}
\\
Das Gasrelais wird ausschließlich, anders als das Buchholzrelais, bei Leistungstransformatoren mit Hermetikabschluss eingesetzt. Es wird an der Außenwand des Transformators am höchsten Punkt montiert, damit das Gasrelais die aufsteigenden Gase einfangen kann. Es können kleinere elektrische und thermische Fehler sowie der Alterungsprozess im Transformator durch die steigende Gasansammlung im Relais festgestellt werden. Bei den kleineren elektrischen Fehlern, handelt es sich um die gleichen Fehler, die schon für das Buchholzrelais beschrieben wurden.
\cite[][S. 31ff]{ieeeGuide}
\\
Beide Relais sind sehr sensible und verlässliche Schutzgeräte die unabhängig von der Windungszahl und bei Transformatoren auch unabhängig von der Position des Laststufenschalters arbeiten.
\cite[][S. 32ff]{abbManual}


\section{Buchholzrelais}
Das Buchholzrelais kann die Gasansammlung, einen Isolierflüssigkeitsverlust und einen Isolierflüssigkeitsstrom anzeigen.

\subsection{Gasansammlung}
Der Alterungsprozess und kleinere Fehler führen im Transformator zur Bildung ungelöster Gase. Diese Gase steigen in der Isolierflüssigkeit nach oben und sammeln sich im Buchholzrelais. Im Buchholzrelais verdrängt die Gasansammlung das Öl und senkt somit den Füllstand. Durch den fallenden Füllstand sinkt der obere Schwimmer. Wird ein bestimmter Füllstand unterschritten, dann schaltet anschließend der Schwimmer einen Schaltkontakt, der ein Alarmsignal ausgibt.
\\
Der untere Schwimmer bleibt dabei unbeeinflusst, da Gasmengen, die das Maximalvolumen überschreiten, über die Rohrleitung zum Ausdehnungsgefäß abströmen.
\cite{embKatalog},
\cite{messkoManual}

\begin{figure}[h]
\centering
\includegraphics[width=8cm]{gasansammlung.png}
\caption{Alarm bei Gasansammlung (Quelle: \cite[][Fig. 5]{embKatalog})}
\end{figure}

\noindent
Anhand der Zusammensetzung der Gasansammlung können Rückschlüsse auf die Ursache der aufgetretenen Fehler gemacht werden. So können Aussagen über den Zustand des Isolationspapiers gemacht werden, welches ausschlaggebend für die Lebensdauer ist. Auch können thermische und elektrische Fehler unterschieden werden. Dies ermöglicht eine frühzeitige Fehlererkennung und verhindert größere Schäden und Ausfälle.
\cite{Gockenbach2007},
\cite[][S. 33]{abbManual}

\subsection{Isolierflüssigkeitsverlust}
Wenn sich ein Leck im Transformator befindet, sinkt das Flüssigkeitsniveau kontinuierlich. Als erstes sinkt der obere Schwimmer, der einen Alarm auslöst. Bei weiterem Isolierflüssigkeitsverlust entleert sich das Ausdehnungsgefäß und die Rohrleitung. Ab einem bestimmten Flüssigkeitsverlust sinkt der untere Schwimmer und löst die Abschaltung des Schutzgeräts aus.
\cite{embKatalog},
\cite{messkoManual},
\cite[][S. 34]{abbManual}

\begin{figure}[h]
\centering
\includegraphics[width=8cm]{isolierfluessigkeitsverlust.png}
\caption{Abschaltung bei Isolierflüssigkeitsverlust (Quelle: \cite[][Fig. 6]{embKatalog})}
\end{figure}

\subsection{Isolierflüssigkeitsströhmung}
Größere Fehler im Transformator führen zu einer schnellen Entstehung von großen Mengen an Gas und Öl-Dampf. Es kann ein Gemisch aus Gas und Öldampf mit mehr als \(50\,cm^3/kWs\) entstehen. Dieses Gemisch verdrängt die umliegende Isolierflüssigkeit und bewirkt eine Druckwelle in Richtung Ausdehnungsgefäß. Der entstandene Flüssigkeitsstrom trifft auf die, im Buchholzrelais verbaute, Stauklappe. Überschreitet der Flüssigkeitsstrom einen bestimmten Schwellenwert, dann klappt die Stauklappe in Richtung des Flüssigkeitsstroms. Das Umklappen schaltet den unteren Schaltkontakt und bewirkt die Abschaltung des Schutzgeräts.
\cite{embKatalog},
\cite[][S. 33]{abbManual}

\begin{figure}[h]
\centering
\includegraphics[width=8cm]{isolierfluessigkeitsstrom.png}
\caption{Abschaltung bei Isolierflüssigkeitsströhmung (Quelle: \cite[][Fig. 7]{embKatalog})}
\end{figure}

\section{Gasrelais}
Wie beim Buchholzrelais verdrängt die steigende Gasansammlung die Isolierflüssigkeit, was zu einem sinkenden Füllstand führt. Dieser sinkende Füllstand führt zum Absenken von Schwimmern die Schaltkontakte schalten. Anders als beim Buchholzrelais können mehr Füllstandniveaus signalisiert werden. Auch ein Isolierflüssigkeitsverlust kann vom Gasrelais festgestellt werden. Es kann allerdings kein Isolierflüssigkeitsstrom vom Gasrelais erkannt werden.
\cite[][S. 32]{ieeeGuide}

\section{Nutzen von Sensoren im Buchholz- und Gasrelais}
Ein großes Problem, dem sich Energieversorgungsunternehmen stellen müssen, ist die weltweite Alterung der Transormatoren. Ein großer Teil der Transformatoren wird schon über ihre ausgelegte Lebensdauer betrieben oder wird sie in den nächsten Jahren erreichen.
\cite{tdworld_transformer_age},
\cite{Koch2008},
\cite{Islam2017}
\\
Laut ABB kann der ungeplante Austausch eines typischen Generatortransformators den Betreiber bis zu 15 Mio. USD kosten.
\cite{abbFit}
\\
Die wichtigsten Einflussfaktoren auf den Alterungsprozess eines Transformators und der damit verbundenen Ausfallrate, sind die Temperatur, die Feuchte als auch der Sauerstoffgehalt im Öl. Es ist deshalb ratsam für den Betreiber diese Werte zu überwachen.
\cite[][S. 12]{Koch2008},
\cite{Hofmann2004}
\\
Zusätzlich ist es wichtig auftretende Fehler oder einen falschen Betrieb frühzeitig erkennen zu können. Für diese frühzeitige Erkennung ist eine Überwachung von weiteren Parametern im laufenden Betrieb des Transformators notwendig. Der Trend geht folglich dahin, dass immer mehr Sensoren in den Transformatoren verbaut werden. Es wäre für die Hersteller und Betreiber von den Transformatoren von Vorteil, wenn  möglichst viele der Sensoren sich im Buchholz- oder Gasrelais befinden. Das Buchholz- oder Gasrelais kann auch bei älteren Transformatoren einfach nachgerüstet oder ausgetauscht werden.
\cite{Gockenbach2007}