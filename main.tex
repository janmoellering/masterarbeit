\documentclass[12pt,a4paper,twoside]{report}
	\makeatletter
	\def\@makechapterhead#1{%
  	%%%%\vspace*{50\p@}% %%% removed!
  	{\parindent \z@ \raggedright \normalfont
    	\ifnum \c@secnumdepth >\m@ne
        	\huge\bfseries \@chapapp\space \thechapter
        	\par\nobreak
        	\vskip 20\p@
    	\fi
    	\interlinepenalty\@M
    	\Huge \bfseries #1\par\nobreak
    	\vskip 40\p@
  	}}
	\def\@makeschapterhead#1{%
  	%%%%%\vspace*{50\p@}% %%% removed!
  	{\parindent \z@ \raggedright
    	\normalfont
    	\interlinepenalty\@M
    	\Huge \bfseries  #1\par\nobreak
    	\vskip 40\p@
  	}}
	\makeatother

\usepackage[utf8]{inputenc}
\usepackage[german]{babel}
\usepackage[T1]{fontenc}
\usepackage{amsmath}
\usepackage{amsfonts}
\usepackage{amssymb}
\usepackage{makeidx}
\usepackage{graphicx}
    \graphicspath{ {./images/} }
\usepackage[hidelinks]{hyperref}
	\hypersetup{
		colorlinks,
		allcolors=black
	}
\usepackage{caption}
\usepackage{subcaption}
\usepackage{pdfpages}
\usepackage{wrapfig}
\usepackage{kpfonts}
\usepackage[a4paper, width=150mm, top=25mm, bottom=25mm, bindingoffset=6mm]{geometry}
\usepackage{fancybox}
\usepackage{fancyvrb}
\usepackage{fancyhdr}
    \pagestyle{fancy}
    \fancyhead{}
    \fancyhead[RO,LE]{Kapitel \thechapter}
\usepackage{array}
\usepackage{tabularx}
\usepackage{booktabs}
\usepackage{longtable}
\usepackage{acronym}
\usepackage{csquotes}
\usepackage[backend=biber, bibencoding=utf8, style=ieee, dashed=false]{biblatex}
\addbibresource{bibliography/bib.bib}

\title{Entwicklung eines Prüfgeräts für Druckschalter}
\author{Jan Möllering}
\date{15.09.2021}


\begin{document}
\emergencystretch 3em

\includepdf{Musterdeckblatt_Master.pdf}

\chapter*{Eidesstattliche Erklärung}
\thispagestyle{empty}
Hiermit erkläre ich, dass ich die vorliegende Arbeit eigenständig und ohne fremde Hilfe angefertigt habe. Textpassagen,
die wörtlich oder dem Sinn nach auf Publikationen oder Vorträgen anderer Autoren beruhen, sind als solche kenntlich gemacht.
\\
\noindent
Die Arbeit wurde bisher keiner anderen Prüfungsbehörde vorgelegt und auch noch nicht veröffentlicht.

\vspace{4cm}

\hspace{2cm} Ort, Datum \hfill Unterschrift \hspace{2cm}

\chapter*{Abstract}
\thispagestyle{empty}
Diese Arbeit beschäftigt sich mit der Entwicklung eines Prüfgeräts für Druckschalter.
\\
\\
\\
\\
\\
This Thesis is about the development of an measurement device for pressure switches.

\setcounter{tocdepth}{2}
\tableofcontents
\thispagestyle{empty}
\newpage
\pagenumbering{Roman}
\addtocontents{toc}{\protect\thispagestyle{empty}}


\addcontentsline{toc}{chapter}{Tabellenverzeichnis}
\listoftables
\newpage

\addcontentsline{toc}{chapter}{Abbildungsverzeichnis}
\listoffigures

\addcontentsline{toc}{chapter}{Abkürzungsverzeichnis}
\chapter*{Abkürzungsverzeichnis}
\input{Abkuerzungsverzeichnis.tex}
\newpage

\chapter{Motivation und Aufgabenstellung}
\pagenumbering{arabic}
\setcounter{page}{1}
\section{Motivation}
\clearpage

\section{Aufgabenstellung}

\textbf{Die Teilaufgaben umfassen:}

\begin{itemize}
	\item Einarbeitung in die ...
	\item Entwicklung der Elektronik für das Prüfgerät
	\begin{itemize}
		\item Unterpunkt 1
		\item Unterpunkt 2
		\item Messen von Stromwerten im \(\mu\)A Bereich
	\end{itemize}
	\item Entwicklung und Programmierung der Software für das Prüfgerät
	\begin{itemize}
		\item Steuerung des Prüfgeräts
		\item Darstellung, Speicherung und Auswertung der Messwerte
		\item Realisierung einer grafischen Oberfläche
	\end{itemize}
	\item Auswahl aller benötigten Komponenten
	\item Aufbau eines funktionsfähigen Prototyps des Prüfgeräts
	\item Erprobung und Testung von Teilkomponenten
\end{itemize}
\clearpage

\section{Detaillierte Anforderungen an das Prüfgeräts}
Im Folgenden ist eine detailliertere Beschreibung der Anforderungen an den Prüfstand aufgelistet.
Diese Anforderungen haben sich in Absprache mit der Abteilung Technik ergeben.

\chapter{Funktionsweise des Buchholz- und Gasrelais}

\chapter{Entwicklung des Prüfgeräts für Druckschalter}

\chapter{Erprobung und Ergebnisse}

\chapter{Zusammenfassung und Ausblick}

\printbibliography

\appendix

\chapter{Datenblätter}
\newpage

\chapter{Entwickelte Module}

\end{document}